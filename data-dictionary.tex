% Table generated by Excel2LaTeX from sheet 'Sheet1'
\begin{table}[H]
  \centering
  \caption{The \texttt{User} table}
    \begin{tabularx}{\textwidth}{lllXl}
    \hline
    Name  & Restrictions & Data Type & Description & Default value \\
    \hline
    UserID PK & NN, U & Integer & Unique user ID & Next integer \\
    Email & NN, U, 255C & String & User's email address &  \\
    First name & NN, 255C & String & User's first name &  \\
    Last name & NN, 255C & String & User's last name &  \\
    IsAdmin & NN    & Integer & Is the user an administrator? & FALSE \\
    \hline
    \end{tabularx}%
  \label{tab:dd:User}%
\end{table}%


% Table generated by Excel2LaTeX from sheet 'Sheet1'
\begin{table}[H]
  \centering
  \caption{The \texttt{Diary} table}
    \begin{tabularx}{\textwidth}{llp{2.5cm}Xl}
    \hline
    Name  & Restrictions & Data Type & Description & Default value \\
    \hline
    DiaryID PK & NN,U  & Integer & Diary entry unique ID & Next integer \\
    UserID & NN    & Integer, FK(User) & User ID of user who created entry &  \\
    Timestamp & NN    & Date  & Date/time of when the entry was created &  \\
    EntryType & NN    & Integer, FK(DiaryType) & The entry type (e.g. Breakfast, Lunch or Dinner) &  \\
    EntryData & NN, 255C & String & The entry contents (what the user wrote for this entry) &  \\
    \hline
    \end{tabularx}%
  \label{tab:dd:Diary}%
\end{table}%


% Table generated by Excel2LaTeX from sheet 'Sheet1'
\begin{table}[H]
  \centering
  \caption{The \texttt{DiaryType} table.}
    \begin{tabularx}{\textwidth}{lllXl}
    \hline
    Name  & Restrictions & Data Type & Description & Default value \\
    \hline
    DiaryTypeID PK & NN, U & Integer & Diary type unique ID & Next integer \\
    Name  & NN, U, 255C & String & The entry type name (e.g. Breakfast, Lunch or Dinner) &  \\
    Description & 255C  & String & An optional description &  \\
    \hline
    \end{tabularx}%
  \label{tab:dd:DiaryType}%
\end{table}%


% Table generated by Excel2LaTeX from sheet 'Sheet1'
\begin{table}[H]
  \centering
  \caption{The \texttt{DailyStats} table.}
    \begin{tabularx}{\textwidth}{llp{2.5cm}Xl}
    \hline
    Name  & Restrictions & Data Type & Description & Default value \\
    \hline
    StatsID PK & NN, U & Integer & DailyStats unique ID & Next integer \\
    UserID & NN    & Integer, FK(User) & User ID for user that this stats entry belongs to &  \\
    Timestamp & NN    & Date  & Day of stats entry &  \\
    Stats & NN    & String & Serialised statistics (e.g. JSON) &  \\
    \hline
    \end{tabularx}%
  \label{tab:dd:DailyStats}%
\end{table}%


% Table generated by Excel2LaTeX from sheet 'Sheet1'
\begin{table}[H]
  \centering
  \caption{The \texttt{Survey} table.}
    \begin{tabularx}{\textwidth}{lllXl}
    \hline
    Name  & Restrictions & Data Type & Description & Default value \\
    \hline
    SurveyID PK & NN, U & Integer & Survey unique ID & Next integer \\
    Description & NN, 255C & String & Survey description &  \\
    Name  & NN, U, 255C & String & The name of the survey &  \\
    \hline
    \end{tabularx}%
  \label{tab:dd:Survey}%
\end{table}%


% Table generated by Excel2LaTeX from sheet 'Sheet1'
\begin{table}[H]
  \centering
  \caption{The \texttt{SurveyQuestion} table.}
    \begin{tabularx}{\textwidth}{llp{2.5cm}Xl}
    \hline
    Name  & Restrictions & Data Type & Description & Default value \\
    \hline
    QuestionID PK & NN, U & Integer & Question unique ID & Next integer \\
    SurveyID & NN    & Integer, FK(Survey) & Survey ID that this question belongs to &  \\
    QuestionText & NN    & String & The question text &  \\
    QuestionType & NN    & String & The question type (e.g. radio, multi-choice, number) &  \\
    Choices & 255C  & String & Comma separated list of choices (if type is suitable) &  \\
    Required & NN    & Boolean & Is this question required? & TRUE \\
    \hline
    \end{tabularx}%
  \label{tab:dd:SurveyQuestion}%
\end{table}%


% Table generated by Excel2LaTeX from sheet 'Sheet1'
\begin{table}[H]
  \centering
  \caption{The \texttt{SurveyResponse} table.}
    \begin{tabularx}{\textwidth}{llp{2.5cm}Xl}
    \hline
    Name  & Restrictions & Data Type & Description & Default value \\
    \hline
    ResponseID PK & NN, U & Integer & Survey response unique ID & Next integer \\
    SurveyID & NN    & Integer, FK(Survey) & ID of survey that this response corresponds to &  \\
    UserID & NN    & Integer, FK(User) & ID of user who completed this response &  \\
    Timestamp & NN    & Date  & Date/time of when this response was completed &  \\
    \hline
    \end{tabularx}%
  \label{tab:dd:SurveyResponse}%
\end{table}%


% Table generated by Excel2LaTeX from sheet 'Sheet1'
\begin{table}[H]
  \centering
  \caption{The \texttt{QuestionResponse} table.}
    \begin{tabularx}{\textwidth}{lllXl}
    \hline
    Name  & Restrictions & Data Type & Description & Default value \\
    \hline
    QResponseID PK & NN, U & Integer & QuestionResponse unique ID & Next integer \\
    ResponseID & NN    & Integer, FK(SurveyResponse) & ID of response that this is part of &  \\
    QuestionID & NN    & Integer, FK(SurveyQuestion) & ID of question that this is answering &  \\
    Answer & NN, 255C & String & The answer to the question &  \\
    \hline
    \end{tabularx}%
  \label{tab:dd:QuestionResponse}%
\end{table}%


% Table generated by Excel2LaTeX from sheet 'Sheet1'
\begin{table}[H]
  \centering
  \caption{A list of abbreviations used}
    \begin{tabular}{ll}
    \hline
    Abbreviation & Description \\
    \hline
    NN    & Not null \\
    U     & Unique \\
    255C  & 255 characters \\
    FK(x) & Foreign Key to x \\
    PK    & Primary Key \\
    \hline
    \end{tabular}%
  \label{tab:abbreviations}%
\end{table}%


